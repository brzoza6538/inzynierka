

Do porównania wyników każdego eksperymentu wykorzystano funkcję ECDF (Empirical Cumulative Distribution Function) oraz wykres średnich wartości w czasie z odchyleniem standardowym, oba wykresy w skali logarytmicznej.
ECDF przedstawia skumulowany ułamek progów błędu osiągniętych przez każdy algorytm w kolejnych punktach w czasie.
Ponieważ wszystkie algorytmy są oceniane względem tych samych progów, metoda dobrze wizualizuje, jak każdy z algorytmów radził sobie w różnych etapach eksperymentu.
Metoda ta uniemożliwia spadek osiągniętych progów, co znacząco poprawia czytelność wykresów