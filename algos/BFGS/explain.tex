



Algorytm BFGS wykorzystuje aproksymacje hesjanu funkcji, pozwalając na skalowanie kroku mając wgląd nie tylko w przebieg funkcji, ale też jej gradientu. Pozwala na dokładniejsze wybory kierunku i wielkości kroków.

L-BFGS-B to rozszerzenie algorytmu BFGS o ograniczenia przestrzeni (Box constraints) oraz pamięci (Limited memory). Przestrzeń rozwiązań jest ograniczona, a hesjan jest aproksymowany na podstawie macierzy ustalonej liczby ostatnich kroków, pracuje na ograniczonej pamięci nie wykorzystując pełnej macierzy hesjanu.


