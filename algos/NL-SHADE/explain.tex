NL-SHADE-RSP-MID to rozszerzenie algorytmu DE, adaptacyjnie ustawiające parametry krzyżowania i mutacji Cr i F, na podstawie populacji oraz archiwum rozwiązań.

NL-SHADE: 

Dla każdego osobnika generowany jest wektor mutant na podstawie trzech losowo wybranych osobników z populacji i potencjalnie archiwum.

Do operatora różnicowego mutacji może zostać wykorzystany punkt wylosowany z archiwum, które jest złożone z rodziców, którzy zostali zastąpieni potomkami z lepszym wynikiem.

Indywidualnie wartości Cr i F są losowane wykorzystując losową wartość z pamięci Cr i F jako centra rozkładów. 

Określamy dla każdego osobnika poprawę wartości funkcji celu względem poprzedniej generacji oraz wartości Cr i F użyte na nich. Na podstawie tych danych, średnią ważoną Lehmera obliczamy najlepszą preferowaną wartość Cr i F dla generacji. Wynik jest zapisywany na jednej z pozycji w pamięci Cr i F.

Pod koniec generacji porównywane są wartości funkcji celu osobników powstałych z mutacji, które wykorzystały archiwum z tymi, które tego nie zrobiły i, na podstawie sukcesu potomków, obliczane jest prawdopodobieństwo użycia archiwum w mutacji dla następnej generacji.

Rozmiar populacji oraz archiwum jest zmieniany w zależności od liczby wywołań funkcji celu w stosunku do limitu.

RSP:

Jeśli punkt nie mieści się w podanej przestrzeni danych, jest generowany ponownie.

MID:

Populacja jest klastrowana względem pozycji w przestrzeni rozwiązań, porównywane są wartości funkcji celu żeby wyznaczyć najlepszy podział, oraz porównać go z pełnym zbiorem żeby określić czy populacja ma zgrupowanie kilku punktów. Wyznaczany jest centroid najlepszego zgrupowania. Jeśli centroid się nie zmienia, oznacza to stagnację i algorytm jest zatrzymywany

