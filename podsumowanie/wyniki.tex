
\subsection*{Algorytmy stochastyczne}
\begin{itemize}
    \item Poruszają się niezależnie w każdym wymiarze, co sprawia, że są wrażliwe na kierunki przekątne w krajobrazie funkcji.
\end{itemize}

\subsection*{Wpływ szumu}
\begin{itemize}
    \item Im bardziej złożony algorytm stochastyczny, tym bardziej wrażliwy jest na szum i chaotyczność funkcji.
\end{itemize}

\subsection*{Charakter algorytmów}
\begin{itemize}
    \item Adam, Adagrad i L-BFGS-B aktualizują parametry jeden wymiar na raz, co powoduje problemy w funkcjach, których minima lokalne tworzą obiekt nierównoległy do osi układu (np. F3 - ``Rotated Bent Cigar Function'').
    \item CMA-ES i inne algorytmy globalne, dzięki rozproszonej populacji, radzą sobie lepiej z kierunkami przekątnymi i wykrywaniem ekstremów w krajobrazie funkcji.
\end{itemize}

\subsection*{Skalowalność}
\begin{itemize}
    \item Algorytmy stochastyczne skalują się liniowo względem liczby wymiarów.
    \item Algorytmy globalne, jak CMA-ES, skalują się potęgowo, co ogranicza ich opłacalność przy dużych wymiarach, mimo ich dokładności dla mniejszych wymiarów.
\end{itemize}

\subsection*{Cechy specyficzne algorytmów}
\begin{itemize}
    \item Zmiana strategii w trakcie działania, np. w nl-shade-rsp-mid, sugeruje nastawienie na długotrwałą eksplorację, w odróżnieniu od CMA-ES, który utrzymuje stałą strategię.
\end{itemize}